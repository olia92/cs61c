%%
\documentclass[11pt,a4paper,english,greek,twoside]{ceid-thesis}

\usepackage{graphicx}
\usepackage{epstopdf}
\usepackage{indentfirst}
\usepackage{verbatim}
\usepackage{amsmath}
\usepackage{amsthm}
\usepackage{amssymb}
\usepackage{latexsym}
\bibliographystyle{static/hellas}
\usepackage{hyphenat}
\usepackage{makeidx}
\addto\captionsgreek{%
  \renewcommand{\indexname}{Ευρετήριο όρων}%
}
\makeindex

% 1.5 spacing
\renewcommand{\baselinestretch}{1.2}

% latin text (and greek text)
\newcommand{\tl}[1]{\textlatin{#1}}
\newcommand{\tg}[1]{\textgreek{#1}}

% typeset short english phrases
\newcommand{\en}[1]{\foreignlanguage{english}{#1}}

% typeset source code
\newcommand{\src}[1]{{\tt\en{#1}}}

% typeset a backslash
\newcommand{\bkslash}{\en{\symbol{92}}}

\newtheorem{definition}{Ορισμός}
\newtheorem{proposition}{Πρόταση}
\newtheorem{theorem}{Θεώρημα}
\newtheorem{corollary}{Συμπέρασμα}
\newtheorem{lemma}{Λήμμα}
\newtheorem{example}{Παράδειγμα}
\newtheorem{remark}{Σημείωση}
\newtheorem{notation}{Συμβολισμός}
\newtheorem{law}{Νόμος}
\renewcommand{\thedefinition}{\arabic{chapter}.\arabic{definition}}
\renewcommand{\theproposition}{\arabic{chapter}.\arabic{proposition}}
\renewcommand{\thetheorem}{\arabic{chapter}.\arabic{theorem}}
\renewcommand{\thecorollary}{\arabic{chapter}.\arabic{corollary}}
\renewcommand{\thelemma}{\arabic{chapter}.\arabic{lemma}}
\renewcommand{\theexample}{\arabic{chapter}.\arabic{example}}
\newcommand{\set}[1]{\left\{#1\right\}}
\newcommand{\To}{\Longrightarrow}
\newcommand{\xml}{\en{XML}}

\selectlanguage{greek}

\hyphenation{ο-ποί-α}

%%%%%%%%%%%%%%%%%%%%%%%%%%%%%%%%%%%%%%%%%%%%%%%%%%%%%
%% THESIS INFO 
%%
%
% Τίτλος Πτυχιακής Εργασίας
	\title{Αναζήτηση και Εξόρυξη Πληροφορίας σε Μεγάλες Βάσεις Αδόμητων Δεδομένων με Μεθόδους Παράλληλης Επεξεργασία}
% "του" ή "της", ανάλογα με το φύλο του σπουδαστή
	\edef\toutis{της}
% Ονοματεπώνυμο σπουδαστή (ΚΕΦΑΛΑΙΑ, γενική πτώση)
	\edef\authorNameCapital{ΤΣΑΜΟΥ ΟΛΥΜΠΙΑΣ}
% Ονοματεπώνυμο σπουδαστή (πεζά, ονομαστική πτώση)
	\author{Ολυμπία Τσάμου}
% Ονοματεπώνυμο Επιβλέποντα Καθηγητή
	\supervisor{Ευάγγελος Δερματάς}
    \edef\supervisorTitle{Καθηγητής}
% Ονοματεπώνυμο Επιβλέποντα Καθηγητή
	\supervisorSecond{}
    \edef\supervisorSecondTitle{**** Τίτλος}
% "Επιβλέπων" ή "Επιβλέπουσα", ανάλογα με το φύλο του Επιβλέποντα Καθηγητή
	\edef\supervisorMaleFemale{Επιβλέποντες}
% Τόπος, μήνας και έτος
	\edef\thesisPlaceDate{Πάτρα, Μάιος 2022}
% Ημερομηνία Εξέτασης
	\edef\examinationDate{19η Μαΐου 2022}
% Έτος Copyright
	\edef\copyrightYear{2022}
% Ονοματεπώνυμο 1ου εξεταστή
	\epitropiF{Θεόδωρος Αντωνακόπουλος}
% Τίτλος 1ου εξεταστή
	\edef\epitropiFTitle{Καθηγητής}
% Ονοματεπώνυμο 2ου εξεταστή
	\epitropiS{****}
% τίτλος 2ου εξεταστή
	\edef\epitropiSTitle{Καθηγητής}
%%%%%%%%%%%%%%%%%%%%%%%%%%%%%%%%%%%%%%%%%%%%%%%%%%%%%


\begin{document}
\selectlanguage{greek}
\maketitle

\frontmatter
% Περίληψη
	%%% Greek Abstract %%%
\thispagestyle{plain}
\begin{center}
\Large
\textbf{Περίληψη}\\ \addcontentsline{toc}{chapter}{Περίληψη}

\vspace{0.9cm}
\large
\textbf{Αναζήτηση και Εξόρυξη Πληροφορίας σε Μεγάλες Βάσεις Αδόμητων Δεδομένων με Μεθόδους Παράλληλης Επεξεργασίας}
    
\vspace{0.4cm}
\begin{minipage}{0.45\textwidth}
    \small
    Φοιτήτρια: 
    \textbf{Ολυμπία Τσάμου}
\end{minipage}
\begin{minipage}{0.45\textwidth}
    \small
    Επιβλέπων:
    \textbf{Ευάγγελος Δερματάς}
\end{minipage}
\\
    \vspace{0.9cm}
\end{center}
%%% Abstract %%%

Εφαρμογή τεχνικών βελτιστοποίησης σε πρόβλημα ταξινόμησης εικόνων κάνοντας χρήση ενός Συνελικτικού Νευρωνικού Δικτύου [\en{cs61c}] \cite{cs61c}

%%%%%%%%%%%%%%%%%%%%%%%%%%
\cleardoublepage
%%% English Abstract %%%
\selectlanguage{english}
\thispagestyle{plain}

\begin{center}
\Large
\textbf{Abstract}\\ \addcontentsline{toc}{chapter}{Abstract}%

\vspace{0.9cm}
\large
\textbf{Data Mining in Large Unstructured Data Bases with Parallel Processing Methods}
    
\vspace{0.4cm}
\begin{minipage}{0.45\textwidth}
    \small
    Student: 
    \textbf{Olympia Tsamou}
\end{minipage}
\begin{minipage}{0.45\textwidth}
    \small
    Supervisor:
    \textbf{Evangellos Dermatas}
\end{minipage}
\\
\vspace{0.9cm}
\end{center}

%%% Abstract %%%
\en{
Application of optimization techniques to an image classification problem using a Convolutional Neural Network [cs61c] \cite{cs61c}
}
%%%%%%%%%%%%%%%%%%%%%%%%%%

% Αφιέρωση
	\thesisDedication{στους γονείς μου}
% Ευχαριστίες
	\begin{acknowledgements}
Θα ήθελα καταρχήν να ευχαριστήσω τον καθηγητή κ. .........
για την επίβλεψη αυτής της διπλωματικής εργασίας [ ] ευχαριστώ ιδιαίτερα τον Δρ.
............ για την καθοδήγησή του και την εξαιρετική
συνεργασία που είχαμε. Τέλος θα ήθελα να ευχαριστήσω τους γονείς
μου για την καθοδήγηση και την ηθική συμπαράσταση που μου
προσέφεραν όλα αυτά τα χρόνια.
\end{acknowledgements}
% Πίνακας Περιεχομένων
	\tableofcontents
% Κατάλογος Σχημάτων
	\listoffigures
% Κατάλογος Πινάκων
	\listoftables

%%%%%%%%%%%%%%%%%%%%%%%%%%%%%%%%%%%%%%%%%%%%%%%%%%%%%
%% INCLUDE YOUR CHAPTERS/SECTIONS HERE
%%
\mainmatter
% Εισαγωγή
	\chapter{Εισαγωγή}

Εισαγωγή.\cite{zoi04}

\section{Ορισμός του Προβλήματος}
Εφαρμογή τεχνικών βελτιστοποίησης σε πρόβλημα ταξινόμησης εικόνων κάνοντας χρήση ενός Συνελικτικού Νευρωνικού Δικτύου \cite{cs61c}


\section{Στόχοι της Διπλωματικής Εργασίας}

Γίνεται έρευνα σχετικά με τις διαφορετικές αρχιτεκτονικές νευρωνικών δικτύων και τις μεθόδους εκπαίδευσης, ωστόσο, μια άλλη κρίσιμη πτυχή των νευρωνικών δικτύων είναι, δεδομένου ενός εκπαιδευμένου δικτύου, η γρήγορη και ακριβής ταξινόμηση εικόνων.
Σε αυτό το \en{project}, δίνετε ένα εκπαιδευμένο νευρωνικό δίκτυο που ταξινομεί εικόνες 32\en{x}32 \en{RGB} σε 10 κατηγορίες. Οι εικόνες ανήκουν στο \en{dataset  CIFAR-10}. Μας δίνετε ο αλγόριθμος εμπρόσθιας διάδοσης του νευρωνικού δικτύου και τα τελικά βάρη του δικτύου. Σκοπός είναι η βελτίωση της ταχύτητας της εμπρόσθιας διάδοσης ώστε να γίνει ταξινόμηση με πιο γρήγορο ρυθμό.

% Κεφάλαια
	\chapter{\selectlanguage{greek}Θεωρητικό Υπόβαθρο}
Κατηγοριοποίηση Εικόνων με χρήση Συνελικτικών Νευρωνικών Δικτύων  
\en{classifying images using a Convolutional Neural Network}

\section{Αναγνώριση Εικόνων}
\subsection{Πώς ένας υπολογιστής αναγνωρίζει εικόνες;}
Η ταξινόμηση εικόνων περιγράφει ένα πρόβλημα στο οποίο σε ένα υπολογιστή δίνεται μία εικόνα και πρέπει να καταλάβει τι απεικονίζει (από ένα σύνολο πιθανών κατηγοριών)
Σήμερα, τα Συνελικτικά Νευρωνικά Δίκτυα (\en{CNNs}) αποτελούν  μια πολύ καλή  προσέγγιση αυτού το προβλήματος. Γενικά, τα νευρωνικά δίκτυα υποθέτουν πως υπάρχει κάποια συνάρτηση από την είσοδο (π.χ. εικόνες) σε μία έξοδο (π.χ. ένα σύνολο κατηγοριών εικόνων). Ενώ οι κλασσικοί αλγόριθμοι προσπαθούν να κωδικοποιήσουν  κάποια πληροφορία του πραγματικού κόσμου στη συνάρτηση τους, τα \en{CNN} μαθαίνουν την συνάρτηση δυναμικά  από ένα σύνολο ταξινομημένων εικόνων (\en{labelled images})—αυτή η διαδικασία ονομάζετε εκπαίδευση. Μόλις καταλήξει σε μια σταθερή συνάρτηση (δηλαδή σε μια προσέγγιση αυτής), μπορεί να εφαρμόσει τη συνάρτηση σε εικόνες που δεν έχει ξαναδεί.
\subsection{Τι μπορεί να κάνει ένα νευρωνικό δίκτυο;}
Ένα νευρωνικό δίκτυο αποτελείται από πολλαπλά επίπεδα.  Κάθε επίπεδο λαμβάνει έναν πολυδιάστατο πίνακα αριθμών ως είσοδο και παράγει έναν άλλο πολυδιάστατο πίνακα αριθμών ως έξοδο (ο οποίος στη συνέχεια γίνεται η είσοδος του επόμενου επιπέδου). Κατά την ταξινόμηση εικόνων, η είσοδος του πρώτου επιπέδου είναι η εικόνα εισόδου  (π.χ. 32\en{x}32\en{x}3 αριθμοί για εικόνες 32\en{x}32 \en{pixel} με 3 κανάλια χρώματος), ενώ η έξοδος του τελευταίου επιπέδου αποτελείται  ένα σύνολο πιθανοτήτων των διαφόρων κατηγοριών (π.χ., 1\en{x}1\en{x}10 αριθμοί αν υπάρχουν 10 κατηγορίες).

\begin{figure}[!ht] \centering 
%\includegraphics[width=\textwidth]{universe}
\includegraphics[width=\textwidth]{static/figures/network.png} \caption{Δομή Συνελικτικού Νευρωνικού Δικτύου}\label{figureCNN}
\end{figure}

Αρχιτεκτονική \en{CNN}~\ref{figureCNN} 

Κάθε επίπεδο έχει ένα σύνολο από βάρη που σχετίζονται με αυτό — αυτά τα βάρη είναι που “μαθαίνει” το νευρωνικό όταν του δοθούν δεδομένα εκπαίδευσης. Ανάλογα με το επίπεδο, τα βάρη έχουν διαφορετικές ερμηνείες, αλλά δεν είναι αντικείμενο μελέτης του συγκεκριμένου \en{project}, φτάνει να γνωρίζουμε ότι κάθε επίπεδο λαμβάνει μία είσοδο, εκτελεί κάποια διεργασία σε αυτή, που εξαρτάται από τα βάρη και παράγει μια έξοδο. Αυτό το βήμα ονομάζεται \en{forward pass}: παίρνουμε μία είσοδο και την προωθούμε στο δίκτυο, παράγοντας το επιθυμητό αποτέλεσμα ως έξοδο. Το \en{forward pass} είναι το μόνο που χρειάζεται για την ταξινόμηση εικόνων σε ένα ήδη εκπαιδευμένο \en{CNN}.
    

Στην πράξη, ένα νευρωνικό δίκτυο αποτελεί μια πολύ απλή μηχανή αναγνώρισης προτύπων (με εξαιρετικά περιορισμένη χωρητικότητα), αλλά μπορεί να είναι αρκετά παράξενο αυτό που καταλήγει να αναγνωρίσει. Για παράδειγμα, κάποιος μπορεί να εκπαιδεύσει ένα νευρωνικό δίκτυο να αναγνωρίζει τη διαφορά μεταξύ “σκύλων” και “λύκων”, και να δουλέψει καλά κοιτώντας το χιόνι και το δάσος στο φόντο των φωτογραφιών με τους λύκους.


	\include{body_matter/chap3}
	\include{body_matter/chap4}
	\include{body_matter/chap5}
	\include{body_matter/chap6}
	\include{body_matter/chap7}
	\include{body_matter/chap8}
	\include{body_matter/chap9}
% Παραρτήματα
	\appendix
	\include{back_matter/appA}
	\include{back_matter/appB}	
	\cleardoublepage
% Βιβλιογραφία - Αναφορές
	\bibliography{back_matter/references}
% Συντομογραφίες - Αρκτικόλεξα - Ακρωνύμια
	\include{back_matter/abbreviations}
% Γλωσσάριο
	\include{back_matter/glossary}
%%%%%%%%%%%%%%%%%%%%%%%%%%%%%%%%%%%%%%%%%%%%%%%%%%%%
\backmatter
% Ευρετήριο Όρων
	\printindex
	\cleardoublepage

%%%%%%%%%%%%%%%%%%
%%%%%%%%%%%%%%%%%%

%% Δημιουργία ετικετών CD:

	\definecdlabeloffsets{0}{-0.65}{0}{0.55} % upper label x offset [cm] (default=0) /  upper label y offset [cm] (default=0) /  lower label x offset [cm] (default=0) /  lower  label y offset [cm] (default=0) -- For Q-Connect KF01579 labels use the following offset values: {0}{-0.65}{0}{0.55}

	\createcdlabel{Αναζήτηση και Εξόρυξη Πληροφορίας \\ σε Μεγάλες Βάσεις Αδόμητων Δεδομένων\\ με Μεθόδους Παράλληλης Επεξεργασίας}{Ολυμπία Ι. Τσάμου}{Μάιος}{2022}{8} % τίτλος πτυχιακής / όνομα συγγραφέα / μήνας / έτος / εύρος περιοχής τίτλου σε cm (προτεινόμενη τιμή: 8) 

%%
%% Δημιουργία εξωφύλλου θήκης CD:

	\createcdcover{Αναζήτηση και Εξόρυξη Πληροφορίας \\ σε Μεγάλες Βάσεις Αδόμητων Δεδομένων\\ με Μεθόδους Παράλληλης Επεξεργασίας}{Ολυμπία Ι. Τσάμου}{Μάιος}{2022}{10} % τίτλος πτυχιακής / όνομα συγγραφέα / μήνας / έτος / εύρος περιοχής τίτλου σε cm (προτεινόμενη τιμή: 10) 

%%
	\pagebreak
	\thispagestyle{empty}
\end{document}
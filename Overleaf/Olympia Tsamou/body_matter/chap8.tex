\chapter{Παράδειγμα Μαθηματικών Σχέσεων -- Εκφράσεων}

\section{Συμπεράσματα}
Τα συστήματα ομότιμων κόμβων, προκειμένου να υποστηρίζουν πιο
εκφραστικές λειτουργίες αναπαράστασης και αναζήτησης δεδομένων,
εξελίχθηκαν στα συστήματα ομότιμων κόμβων τα οποία βασίζονται στις
τεχνολογίες του Σημασιολογικού Ιστού για την αναπαράσταση των
δεδομένων μέσω σχημάτων που τα περιγράφουν (\en{Schema-based
peer-to-peer systems}).

Στα συστήματα αυτά κάθε \en{$\displaystyle y=\int_0^1f(x)dx$} \en{$y=\int_0^1f(x)dx$} κόμβος χρησιμοποιεί ένα σχήμα για την \en{$\displaystyle \sum_{i=0}^{100}a_i$}
αναπαράσταση των δεδομένων του. Όμως σε ένα σύστημα ομότιμων
κόμβων, κάθε κόμβος έχει διαφορετικές απαιτήσεις αναπαράστασης
δεδομένων. Επομένως πρέπει να υπάρχει ευελιξία στην επιλογή \en{$\displaystyle \frac{1}{1+x^2}$}
σχήματος. Τα συστήματα που έχουν προταθεί μέχρι τώρα και παρέχουν
αυτή την ευελιξία, για να είναι δυνατή η αναζήτηση πληροφορίας,
απαιτούν την ύπαρξη κανόνων αντιστοίχισης μεταξύ των σχημάτων με
βάση τους οποίους να μετασχηματίζονται οι ερωτήσεις. Όμως δεν
υποστηρίζεται ακόμα αυτόματη δημιουργία και δυναμική ανανέωση των
κανόνων, που είναι απαραίτητα για τα συστήματα ομότιμων κόμβων.
\begin{equation}
	y=\int_0^1f(x)dx
	\label{equation08.01}
\end{equation}

Η συνεισφορά της (\ref{equation08.01}) παρούσας διπλωματικής εργασίας έχει δύο σκέλη. Το
πρώτο αφορά τη δημιουργία ενός πλήρους συστήματος ομότιμων κόμβων
βασισμένο σε σχήματα \en{RDF} το οποίο παρέχει: (α) την υποδομή
για την επικοινωνία των κόμβων,(β) μηχανισμό δημιουργίας σχήματος,
(γ) μηχανισμό ενσωμάτωσης σχεσιακών δεδομένων στο σχήμα με τη
χρήση αντιστοιχίσεων που δημιουργεί ο χρήστης με τη βοήθεια
ειδικής διαπροσωπείας, (δ) ευέλικτη διαπροσωπεία χρήστη για τη
διατύπωση ερωτημάτων και (ε) μηχανισμό απάντησης και επεξεργασίας
ερωτήσεων.

Το δεύτερο σκέλος αφορά το γεγονός ότι το συγκεκριμένο σύστημα
προσφέρει μια σχετική ευελιξία ως προς την επιλογή του σχήματος
από τον κάθε κόμβο, ενώ ταυτόχρονα δίνει τη δυνατότητα
μετασχηματισμού ερωτήσεων χωρίς τη χρήση κανόνων αντιστοίχισης.
Συγκεκριμένα, τα σχήματα των κόμβων αποτελούν
υποσύνολα$-$όψεις$($\en{views}) ενός βασικού σχήματος που
ονομάζεται καθολικό σχήμα. Εκμεταλλευόμενοι λοιπόν το γεγονός ότι
τα σχήματα αυτά είναι συμβατά μεταξύ τους, έχουμε τη δυνατότητα
ελέγχου της ικανοποιησιμότητας μιας ερώτησης και μετατροπής της
όπου χρειάζεται, χρησιμοποιώντας τόσο το σχήμα του κόμβου όσο και
το καθολικό σχήμα.

Συμπερασματικά το σύστημα που αναπτύχθηκε στα πλαίσια αυτής της
διπλωματικής είναι ένα πλήρες σύστημα ομότιμων κόμβων βασισμένο σε
σχήματα, το οποίο καθιστά δυνατή την αναζήτηση της πληροφορίας με
ένα διαφορετικό τρόπο απ' ότι τα προϋπάρχοντα  συστήματα.

\section{Μελλοντικές Επεκτάσεις}
Το σύστημα που αναπτύχθηκε στα πλαίσια αυτής της διπλωματικής
εργασίας θα μπορούσε να βελτιωθεί και να επεκταθεί περαιτέρω,
τουλάχιστον ως προς τρεις κατευθύνσεις. Συγκεκριμένα, αναφέρονται
τα ακόλουθα:

\begin{itemize}
\item Ενσωμάτωση διαδικασίας επιλογής σχήματος με βάση το οποίο ο
κόμβος θα συμμετέχει στο σύστημα. Έτσι όπως έχει σχεδιαστεί το
σύστημα, κάθε κόμβος έχει τη δυνατότητα να δημιουργήσει πολλά
σχήματα και να αποθηκεύσει δεδομένα σε περισσότερα από ένα. Ως
σχήμα του κόμβου (με βάση το οποίο απαντάει τις ερωτήσεις),
θεωρείται το τελευταίο στο οποίο αποθήκευσε δεδομένα. Η δυνατότητα
επιλογής θα του παρείχε περισσότερη ευελιξία.
\item Δυνατότητα αντιστοίχισης δεδομένων τα οποία να μην είναι
αποθηκευμένα σε βάση δεδομένων αλλά σε αρχεία. Η αποδέσμευση από
τη βάση δεδομένων θα έκανε το σύστημα πιο εύκολο στην εγκατάσταση
και τη χρήση.
\item Αξιολόγηση του συστήματος ως προς τη συμπεριφορά του αν
συμμετέχει σε αυτό μεγάλος αριθμός κόμβων \en{(scalability
testing)} και αν χρησιμοποιηθεί ένα πολύ μεγάλο καθολικό σχήμα. H
αξιολόγηση αυτή αφορά την ταχύτητα με την οποία ένας κόμβος
παίρνει απαντήσεις σε μια ερώτηση καθώς και την ποιότητα των
απαντήσεων.
\end{itemize}
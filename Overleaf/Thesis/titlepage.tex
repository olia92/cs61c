\begin{titlepage}
%%% COVER %%%
\newgeometry{left=2cm,right=2cm,bottom=2cm,top=2.5cm}
    \begin{minipage}{4cm}
    \centering
        \includegraphics[width=4cm]{PatrasLogo}
    \end{minipage}
    \hspace{0.25cm}
    \begin{minipage}{0.6\textwidth}
        \large
        \textbf{ΠΑΝΕΠΙΣΤΗΜΙΟ ΠΑΤΡΩΝ}\\%[4pt]
        \normalsize
        \textbf{ΤΜΗΜΑ ΗΛΕΚΤΡΟΛΟΓΩΝ ΜΗΧΑΝΙΚΩΝ ΚΑΙ ΤΕΧΝΟΛΟΓΙΑΣ ΥΠΟΛΟΓΙΣΤΩΝ}\\%[14pt]
     ΤΟΜΕΑΣ ΤΗΛΕΠΙΚΟΙΝΩΝΙΩΝ ΚΑΙ ΤΕΧΝΟΛΟΓΙΑΣ ΠΛΗΡΟΦΟΡΙΑΣ\\    
    \end{minipage}
    \vspace{0.5cm}
    \hrule
    \hfill
    
    \begin{center}
        \vspace*{2cm}

        \LARGE
        \textbf{Αναζήτηση και Εξόρυξη Πληροφορίας σε Μεγάλες Βάσεις Αδόμητων Δεδομένων με Μεθόδους Παράλληλης Επεξεργασίας}
        
        \vspace{2cm}
        \Large
        ΔΙΠΛΩΜΑΤΙΚΗ ΕΡΓΑΣΙΑ
        
        \vspace{0.5cm}
        \LARGE
        \textbf{ΟΛΥΜΠΙΑΣ ΤΣΑΜΟΥ}
        
        \vspace{3.5cm}
        \vfill
        
        \flushleft

       \large \hspace{0.8cm} \textit{Επιβλέπων:} \textbf{Ευάγγελος Δερματάς}\\
     \centering
    \vspace{1cm}
    \vfill

       Πάτρα, Δεκέμβριος 2022
        
    \end{center}
    \restoregeometry
%%%%%%%%%%%%%%%%%%%%%%%%%%%%%%%%%%%%%%%%%%%%%%%%%%%

%%% Copyright Page %%%
\cleardoublepage
    \thispagestyle{empty}
    \vspace*{\fill}
    \begin{minipage}{\textwidth}
        
    \small
        Πανεπιστήμιο Πατρών, Τμήμα Ηλεκτρολόγων Μηχανικών και Τεχνολογίας Υπολογιστών.\\[10pt]
        Ολυμπία Τσάμου\\[10pt]
        \en{\copyright } 2022 – Με την επιφύλαξη παντός δικαιώματος\\[10pt]
        Το σύνολο της εργασίας αποτελεί πρωτότυπο έργο, παραχθέν από την  Ολυμπία Τσάμου, και δεν παραβιάζει δικαιώματα τρίτων καθ’ οιονδήποτε τρόπο. Αν η εργασία περιέχει υλικό, το οποίο δεν έχει παραχθεί από την ίδια, αυτό είναι ευδιάκριτο και αναφέρεται ρητώς εντός του κειμένου της εργασίας ως προϊόν εργασίας τρίτου, σημειώνοντας με παρομοίως σαφή τρόπο τα στοιχεία ταυτοποίησής του, ενώ παράλληλα βεβαιώνει πως στην περίπτωση χρήσης αυτούσιων γραφικών αναπαραστάσεων, εικόνων, γραφημάτων κ.λπ., έχει λάβει τη χωρίς περιορισμούς άδεια του κατόχου των πνευματικών δικαιωμάτων για την συμπερίληψη και επακόλουθη δημοσίευση του υλικού αυτού.
    \end{minipage}
%%%%%%%%%%%%%%%%%%%%%%%%%%%%%%%%%%%%%%%%%%%%%%%%%%%%

%%% CERTIFICATION %%%
\pagenumbering{gobble}
\cleardoublepage
% \newgeometry{left=2cm,right=2cm}
\begin{center}
\thispagestyle{empty}
        \Large
    \textbf{ΠΙΣΤΟΠΟΙΗΣΗ}\\         
    
    \vfill
    \large
    Πιστοποιείται ότι η Διπλωματική Εργασία με τίτλο\\
        \vfill
        
    \Large  
    \textbf{Αναζήτηση και Εξόρυξη Πληροφορίας σε Μεγάλες Βάσεις Αδόμητων Δεδομένων με Μεθόδους Παράλληλης Επεξεργασίας}\\
        \vfill
        \large
            της φοιτήτριας του Τμήματος Ηλεκτρολόγων Μηχανικών και Τεχνολογίας Υπολογιστών\\
        \vfill
            
            \large{\textbf{ΟΛΥΜΠΙΑΣ ΤΣΑΜΟΥ} ΤΟΥ \textbf{ΙΩΑΝΝΗ}}\\[7pt]
            Αριθμός Μητρώου: 227644\\
        \vfill
            
            Παρουσιάστηκε δημόσια στο Τμήμα Ηλεκτρολόγων Μηχανικών και Τεχνολογίας Υπολογιστών στις\\
        \vfill

            ………/………/………\\
        \vfill

        και εξετάστηκε από την ακόλουθη εξεταστική επιτροπή:\\[8pt]
        
        Ευάγγελος Δερματάς, Αναπληρωτής Καθηγητής, Τμήμα Μηχανικών Η/Υ και Πληροφορικής (επιβλέπων)\\
        Θεόδωρος Αντωνακόπουλος, Καθηγητής, Τμήμα Ηλεκτρολόγων Μηχανικών και Τεχνολογίας Υπολογιστών (μέλος επιτροπής)\\

        \vfill
        \begin{center}
        \begin{tabular}{c c c}
          Ο Επιβλέπων &  & Ο Διευθυντής του Τομέα    \\
           &  &    \\[28pt]
          Ευάγγελος Δερματάς  &  & Κυριάκος Σγάρμπας   \\[8pt]
          Αναπληρωτής Καθηγητής &  & Καθηγητής 
        \end{tabular}
        \end{center}

    \end{center}
    % \restoregeometry

%%%%%%%%%%%%%%%%%
    
\end{titlepage}